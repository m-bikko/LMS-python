\section*{Appendix A: System Screenshots}
\addcontentsline{toc}{section}{Appendix A: System Screenshots}

This appendix provides screenshots of the key interfaces in the subscription-based Learning Management System.

\begin{figure}[h]
\centering
% Insert screenshot of admin dashboard here
\caption{Administrator Dashboard}
\label{fig:admin-dashboard}
\end{figure}

\begin{figure}[h]
\centering
% Insert screenshot of teacher dashboard here
\caption{Teacher Dashboard}
\label{fig:teacher-dashboard-appendix}
\end{figure}

\begin{figure}[h]
\centering
% Insert screenshot of student dashboard here
\caption{Student Dashboard}
\label{fig:student-dashboard}
\end{figure}

\begin{figure}[h]
\centering
% Insert screenshot of course creation form here
\caption{Course Creation Form with Subscription Plan Selection}
\label{fig:course-creation-form}
\end{figure}

\begin{figure}[h]
\centering
% Insert screenshot of payment verification interface here
\caption{Admin Payment Verification Interface}
\label{fig:payment-verification-interface}
\end{figure}

\begin{figure}[h]
\centering
% Insert screenshot of student course enrollment here
\caption{Student Course Enrollment with Payment}
\label{fig:student-enrollment}
\end{figure}

\begin{figure}[h]
\centering
% Insert screenshot of course content management here
\caption{Course Content Management Interface}
\label{fig:content-management}
\end{figure}

\begin{figure}[h]
\centering
% Insert screenshot of course browsing page here
\caption{Course Browsing Interface with Price Display}
\label{fig:course-browsing}
\end{figure}

\section*{Appendix B: Database Schema}
\addcontentsline{toc}{section}{Appendix B: Database Schema}

This appendix provides the complete database schema for the subscription-based Learning Management System, illustrating the relationships between all entities in the system.

\begin{figure}[h]
\centering
% Insert detailed database schema diagram here
\caption{Complete Database Schema}
\label{fig:database-schema}
\end{figure}

\section*{Appendix C: User Guide}
\addcontentsline{toc}{section}{Appendix C: User Guide}

This section provides a brief user guide for the key roles in the system.

\subsection*{C.1 Administrator Guide}

\subsubsection*{C.1.1 Verifying Teacher Payments}

To verify a teacher's subscription payment:

\begin{enumerate}
    \item Log in with administrator credentials
    \item Navigate to "Manage Courses"
    \item Locate the course pending approval
    \item Click "View Receipt" to examine the payment receipt
    \item If the receipt is valid, click "Verify Payment"
    \item Once payment is verified, you can approve the course
\end{enumerate}

\subsubsection*{C.1.2 Managing Subscription Plans}

To manage subscription plans:

\begin{enumerate}
    \item Log in with administrator credentials
    \item Navigate to "System Settings"
    \item Select "Subscription Plans"
    \item From here, you can view, edit, add, or remove subscription plans
\end{enumerate}

\subsection*{C.2 Teacher Guide}

\subsubsection*{C.2.1 Creating a Course with Subscription Plan}

To create a new course:

\begin{enumerate}
    \item Log in with teacher credentials
    \item Navigate to "Create Course"
    \item Fill in course details (title, description, category)
    \item Set the enrollment price for students
    \item Select the appropriate subscription plan based on expected student numbers
    \item Upload a payment receipt for your subscription
    \item Submit the course for approval
\end{enumerate}

\subsubsection*{C.2.2 Verifying Student Payments}

To verify student enrollment payments:

\begin{enumerate}
    \item Log in with teacher credentials
    \item Navigate to "My Courses"
    \item Select the relevant course
    \item Go to "Students" tab
    \item View pending enrollments
    \item Click "View Receipt" to examine the payment
    \item Click "Verify Payment" to approve the enrollment
\end{enumerate}

\subsection*{C.3 Student Guide}

\subsubsection*{C.3.1 Enrolling in a Course with Payment}

To enroll in a course:

\begin{enumerate}
    \item Log in with student credentials
    \item Browse available courses
    \item Select a course of interest
    \item Click "Enroll"
    \item Note the enrollment price
    \item Make payment using your preferred method
    \item Upload the payment receipt
    \item Wait for teacher verification (usually within 24 hours)
\end{enumerate}

\subsubsection*{C.3.2 Accessing Course Content}

To access course content after enrollment:

\begin{enumerate}
    \item Log in with student credentials
    \item Navigate to "My Courses"
    \item Select the course
    \item Navigate through modules using the left sidebar
    \item Complete assignments and tests as required
    \item Track your progress through the course dashboard
\end{enumerate}

\section*{Appendix D: Implementation Code Samples}
\addcontentsline{toc}{section}{Appendix D: Implementation Code Samples}

This appendix provides additional code samples from the implementation that were not included in the main text.

\subsection*{D.1 User Authentication and Registration}

\begin{lstlisting}[language=Python, caption=User Registration Implementation]
@auth_bp.route('/register', methods=['GET', 'POST'])
def register():
    if current_user.is_authenticated:
        return redirect(url_for('index'))
        
    form = RegistrationForm()
    if form.validate_on_submit():
        # Check if email already exists
        existing_user = User.query.filter_by(email=form.email.data).first()
        if existing_user:
            flash('Email already registered. Please use a different email.', 'danger')
            return render_template('auth/register.html', title='Register', form=form)
            
        # Create new user
        user = User(
            email=form.email.data,
            first_name=form.first_name.data,
            last_name=form.last_name.data,
            role='student'  # Default role is student
        )
        user.set_password(form.password.data)
        
        db.session.add(user)
        db.session.commit()
        
        flash('Registration successful! You can now log in.', 'success')
        return redirect(url_for('auth.login'))
        
    return render_template('auth/register.html', title='Register', form=form)
\end{lstlisting}

\subsection*{D.2 Module and Material Management}

\begin{lstlisting}[language=Python, caption=Module Creation and Management]
@teacher_bp.route('/course/<int:course_id>/modules/create', methods=['GET', 'POST'])
@login_required
@teacher_required
def create_module(course_id):
    course = Course.query.get_or_404(course_id)
    
    # Ensure teacher owns the course
    if course.teacher_id != current_user.id:
        flash('Access denied: You do not own this course.', 'danger')
        return redirect(url_for('teacher.dashboard'))
    
    form = ModuleForm()
    
    if form.validate_on_submit():
        # Get the highest order value to place this module at the end
        highest_order = db.session.query(func.max(Module.order)).filter_by(course_id=course_id).scalar()
        new_order = (highest_order or 0) + 1
        
        module = Module(
            title=form.title.data,
            description=form.description.data,
            course_id=course_id,
            order=new_order
        )
        
        db.session.add(module)
        db.session.commit()
        
        flash('Module created successfully!', 'success')
        return redirect(url_for('teacher.view_module', module_id=module.id))
    
    return render_template('teacher/create_module.html', form=form, course=course)
\end{lstlisting}

\subsection*{D.3 Test and Assessment Implementation}

\begin{lstlisting}[language=Python, caption=Test Creation and Question Management]
@teacher_bp.route('/module/<int:module_id>/tests/create', methods=['GET', 'POST'])
@login_required
@teacher_required
def create_test(module_id):
    module = Module.query.get_or_404(module_id)
    course = module.course
    
    # Ensure teacher owns the course
    if course.teacher_id != current_user.id:
        flash('Access denied: You do not own this course.', 'danger')
        return redirect(url_for('teacher.dashboard'))
    
    form = TestForm()
    
    if form.validate_on_submit():
        test = Test(
            title=form.title.data,
            description=form.description.data,
            module_id=module_id,
            passing_score=form.passing_score.data,
            time_limit=form.time_limit.data
        )
        
        db.session.add(test)
        db.session.commit()
        
        flash('Test created successfully! Now add questions to your test.', 'success')
        return redirect(url_for('teacher.edit_test', test_id=test.id))
    
    return render_template('teacher/create_test.html', form=form, module=module, course=course)
\end{lstlisting}

\subsection*{D.4 Responsive CSS Implementation}

\begin{lstlisting}[language=CSS, caption=Responsive Dashboard CSS]
/* Dashboard widgets responsive layout */
.dashboard-widgets {
  display: grid;
  grid-template-columns: 1fr;
  gap: var(--spacing-md);
}

.widget {
  background-color: var(--bg-white);
  border-radius: var(--radius-lg);
  box-shadow: var(--shadow-md);
  padding: var(--spacing-lg);
  transition: transform var(--transition-normal);
  position: relative;
  overflow: hidden;
  display: flex;
  flex-direction: column;
}

/* Media queries for responsive grid */
@media (min-width: 576px) {
  .dashboard-widgets {
    grid-template-columns: repeat(2, 1fr);
  }
}

@media (min-width: 992px) {
  .dashboard-widgets {
    grid-template-columns: repeat(3, 1fr);
  }
}

@media (min-width: 1200px) {
  .dashboard-widgets {
    grid-template-columns: repeat(4, 1fr);
  }
}

/* Mobile optimizations for tables */
@media (max-width: 768px) {
  .table-responsive {
    display: block;
    width: 100%;
    overflow-x: auto;
  }
  
  .table-responsive table {
    min-width: 500px;
  }
  
  /* Convert tables to cards on mobile */
  .mobile-card-view tr {
    display: block;
    margin-bottom: 1.5rem;
    border: 1px solid var(--border-color);
    border-radius: var(--radius-md);
  }
  
  .mobile-card-view td {
    display: block;
    text-align: right;
    position: relative;
    padding-left: 50%;
  }
  
  .mobile-card-view td:before {
    content: attr(data-label);
    position: absolute;
    left: 1rem;
    font-weight: bold;
    text-align: left;
  }
}
\end{lstlisting}