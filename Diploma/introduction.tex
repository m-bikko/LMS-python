\section{Introduction}

In the rapidly evolving landscape of education technology, Learning Management Systems (LMS) have become essential tools for delivering educational content in flexible and accessible ways. Traditional LMS platforms often focus on institutional needs, where educational organizations pay for platform access and then distribute content to their students \cite{aldiab2019}. However, this model does not adequately address the needs of independent educators and learners outside formal institutional settings.

\subsection{Background and Context}

The global e-learning market has experienced extraordinary growth, particularly accelerated by the COVID-19 pandemic, with a projected market size of over \$370 billion by 2026 \cite{markets2021}. This growth reflects the increasing acceptance of online education as a viable alternative to traditional classroom learning. Concurrently, there has been a rise in independent educators seeking platforms to monetize their expertise and knowledge \cite{kim2020}.

Traditional LMS platforms such as Moodle, Canvas, and Blackboard primarily serve educational institutions with large budgets \cite{turnbull2019}. Meanwhile, content marketplace platforms like Udemy and Coursera offer distribution channels for independent educators but impose significant revenue sharing models and restrictive policies that may limit educator autonomy \cite{deng2019}.

\subsection{Problem Statement}

Despite the abundance of educational technology solutions, there remains a significant gap in platforms that offer:

\begin{itemize}
    \item Flexible subscription models for teachers based on their expected student capacity
    \item Direct payment systems with transparent fee structures
    \item Verification mechanisms to ensure legitimate transactions
    \item Full content ownership and control for educators
    \item Affordable access options for students
\end{itemize}

Many independent educators struggle to find platforms that balance fair compensation with reasonable platform fees. Similarly, students often face high course prices due to platform commissions or must commit to expensive subscriptions for access to limited content of interest.

\subsection{Project Objectives}

This diploma project aims to address these challenges by designing and implementing a subscription-based LMS with the following primary objectives:

\begin{enumerate}
    \item Develop a platform that enables teachers to publish courses based on subscription tiers linked to student capacity limits
    \item Implement a system where students can purchase access to individual courses at prices set by the educators
    \item Create secure verification mechanisms for payment receipts to maintain financial integrity
    \item Design an intuitive, responsive user interface that enhances the learning experience
    \item Establish a sustainable business model that fairly compensates all stakeholders
\end{enumerate}

\subsection{Significance of the Project}

This project contributes to the educational technology landscape by proposing a novel approach to LMS monetization that centers on teacher empowerment and student choice. By allowing educators to select subscription plans based on their expected audience size and enabling them to set their own course prices, the platform encourages fair compensation for educational content creation.

For students, the system provides access to quality education without forcing them into bundled subscriptions or platform-wide commitments. This a-la-carte approach to educational content access aligns with contemporary consumer preferences for personalized and flexible services \cite{hwang2018}.

\subsection{Scope and Limitations}

The scope of this project encompasses:

\begin{itemize}
    \item Design and implementation of core LMS functionality (user registration, course creation, content management, enrollment)
    \item Development of subscription tier management for teachers
    \item Implementation of payment verification systems
    \item Creation of basic learning tools (materials, assignments, tests, grades)
    \item Development of communication features between students and teachers
\end{itemize}

However, the project has several limitations:

\begin{itemize}
    \item The platform does not handle payment processing directly, instead relying on verification of external payment receipts
    \item Advanced learning analytics and AI-driven personalization features are beyond the current scope
    \item Integration with third-party educational tools is limited
    \item Real-time collaboration features are not implemented in the initial version
\end{itemize}

\subsection{Thesis Structure}

The remainder of this thesis is organized as follows:

\begin{itemize}
    \item Chapter 2 presents a comprehensive literature review of existing LMS platforms, business models in educational technology, and relevant technical approaches.
    \item Chapter 3 outlines the methodology employed in the research and development process.
    \item Chapter 4 describes the system design, including architecture, database design, and user interface considerations.
    \item Chapter 5 details the implementation process, focusing on key technical challenges and solutions.
    \item Chapter 6 presents evaluation results and discusses the system's performance against project objectives.
    \item Chapter 7 concludes the thesis with a summary of contributions and suggestions for future work.
\end{itemize}