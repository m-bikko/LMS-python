\section{Implementation}

This chapter outlines the implementation of the subscription-based Learning Management System, focusing on key components and technical solutions.

\subsection{Development Environment and Database Implementation}

The system was developed using Python 3.10 with Flask framework in a virtual environment managed via venv. Key dependencies included Flask-SQLAlchemy for database ORM, Flask-Login for authentication, and Flask-WTF for form handling.

\subsubsection{Database Models}

The TeacherSubscriptionPlan model represents the core innovation of the system:

\begin{lstlisting}[language=Python, caption=TeacherSubscriptionPlan Model, label=lst:subscription-model]
class TeacherSubscriptionPlan(db.Model):
    __tablename__ = 'teacher_subscription_plans'
    
    id = db.Column(db.Integer, primary_key=True)
    name = db.Column(db.String(50), nullable=False)
    max_students = db.Column(db.Integer, nullable=False)
    price_per_month = db.Column(db.Integer, nullable=False)
    description = db.Column(db.String(255))
    
    # Relationships
    courses = db.relationship('Course', backref='subscription_plan')
    
    def is_within_capacity(self, teacher_id):
        """Check if a teacher is within their subscription capacity"""
        from .course import Course, Enrollment
        
        # Count active courses by this teacher using this plan
        courses = Course.query.filter_by(
            teacher_id=teacher_id,
            subscription_plan_id=self.id,
            is_active=True
        ).all()
        
        # Count unique students enrolled across these courses
        student_ids = set()
        for course in courses:
            enrollments = Enrollment.query.filter_by(
                course_id=course.id,
                payment_verified=True
            ).all()
            student_ids.update(e.student_id for e in enrollments)
        
        return len(student_ids) <= self.max_students
\end{lstlisting}

\subsubsection{Database Migrations}

Schema evolution was handled via a migration script that creates necessary tables and adds payment-related columns incrementally:

\begin{lstlisting}[language=Python, caption=Migration Script Excerpt, label=lst:migration]
def migrate_db():
    """Run database migrations to update schema"""
    conn = sqlite3.connect('instance/lms.db')
    cursor = conn.cursor()
    
    # Create subscription plans table if needed
    cursor.execute("""
    SELECT name FROM sqlite_master
    WHERE type='table' AND name='teacher_subscription_plans'
    """)
    
    if not cursor.fetchone():
        cursor.execute('''
        CREATE TABLE teacher_subscription_plans (
            id INTEGER PRIMARY KEY AUTOINCREMENT,
            name VARCHAR(50) NOT NULL,
            max_students INTEGER NOT NULL,
            price_per_month INTEGER NOT NULL,
            description TEXT
        )
        ''')
        
        # Add default plans
        subscription_plans = [
            ("Standard - Up to 10 students", 10, 20000, 
             "Basic plan for small classes"),
            ("Pro - Up to 30 students", 30, 35000, 
             "Professional plan for medium classes"),
            ("Premium - Up to 100 students", 100, 70000, 
             "Premium plan for large classes"),
            ("Enterprise - Unlimited", 10000, 100000, 
             "Enterprise plan for unlimited students")
        ]
        
        cursor.executemany(
            "INSERT INTO teacher_subscription_plans (name, max_students, price_per_month, description) VALUES (?, ?, ?, ?)",
            subscription_plans
        )
\end{lstlisting}

\subsection{Authentication and Authorization}

Security was implemented using Flask-Login for session management with a role-based access control system:

\begin{lstlisting}[language=Python, caption=Role-Based Access Control, label=lst:rbac]
def admin_required(f):
    """Restrict access to admin users"""
    @wraps(f)
    def decorated_function(*args, **kwargs):
        if not current_user.is_authenticated or not current_user.is_admin():
            flash('Access restricted. Admin privileges required.', 'danger')
            return redirect(url_for('auth.login'))
        return f(*args, **kwargs)
    return decorated_function

# Applied to routes
@admin_bp.route('/dashboard')
@login_required
@admin_required
def dashboard():
    return render_template('admin/dashboard.html')
\end{lstlisting}

\subsection{Subscription and Payment Features}

\subsubsection{Teacher Subscription Management}

Course creation includes subscription plan selection and payment receipt upload:

\begin{lstlisting}[language=Python, caption=Course Creation with Subscription, label=lst:subscription-management]
@teacher_bp.route('/create-course', methods=['GET', 'POST'])
@login_required
@teacher_required
def create_course():
    form = CourseCreationForm()
    
    # Populate subscription plan choices
    subscription_plans = TeacherSubscriptionPlan.query.all()
    form.subscription_plan.choices = [
        (p.id, f"{p.name} - {p.price_per_month} KZT/month") 
        for p in subscription_plans
    ]
    
    if form.validate_on_submit():
        course = Course(
            title=form.title.data,
            description=form.description.data,
            teacher_id=current_user.id,
            category_id=form.category.data,
            enrollment_price=form.enrollment_price.data,
            subscription_plan_id=form.subscription_plan.data
        )
        
        # Handle payment receipt upload
        if form.payment_receipt.data:
            filename = secure_filename(form.payment_receipt.data.filename)
            timestamp = int(time.time())
            receipt_filename = f"{course.id}_{timestamp}_{filename}"
            filepath = os.path.join(
                current_app.config['UPLOAD_FOLDER'], 
                'payments', 
                receipt_filename
            )
            form.payment_receipt.data.save(filepath)
            course.payment_receipt_path = f"uploads/payments/{receipt_filename}"
        
        db.session.add(course)
        db.session.commit()
        return redirect(url_for('teacher.dashboard'))
\end{lstlisting}

\subsubsection{Student Enrollment with Payment}

Students enroll in courses by uploading payment receipts:

\begin{lstlisting}[language=Python, caption=Student Enrollment, label=lst:enrollment]
@student_bp.route('/enroll/<int:course_id>', methods=['GET', 'POST'])
@login_required
@student_required
def enroll_course(course_id):
    course = Course.query.get_or_404(course_id)
    form = EnrollmentForm()
    
    if form.validate_on_submit():
        enrollment = Enrollment(
            student_id=current_user.id,
            course_id=course.id,
            payment_amount=course.enrollment_price,
            payment_date=datetime.utcnow()
        )
        
        # Handle payment receipt upload
        if form.payment_receipt.data:
            filename = secure_filename(form.payment_receipt.data.filename)
            timestamp = int(time.time())
            receipt_filename = f"enrollment_{course.id}_{current_user.id}_{timestamp}_{filename}"
            filepath = os.path.join(
                current_app.config['UPLOAD_FOLDER'], 
                'enrollment_receipts', 
                receipt_filename
            )
            form.payment_receipt.data.save(filepath)
            enrollment.payment_receipt_path = f"uploads/enrollment_receipts/{receipt_filename}"
        
        db.session.add(enrollment)
        db.session.commit()
        return redirect(url_for('student.dashboard'))
\end{lstlisting}

\subsection{User Interface Implementation}

The UI was built using Jinja2 templates with Bootstrap 4, implementing responsive design and modal dialogs for viewing certificates and receipts:

\begin{lstlisting}[language=HTML, caption=Payment Receipt Modal, label=lst:modal-html]
<!-- Button trigger for modal -->
<button type="button" class="btn btn-sm btn-info" data-toggle="modal" data-target="#paymentModal{{ course.id }}">
    <i class="fas fa-receipt mr-1"></i> View Receipt
</button>

<!-- Payment Receipt Modal -->
<div class="modal fade" id="paymentModal{{ course.id }}" tabindex="-1" role="dialog" aria-labelledby="paymentModalLabel{{ course.id }}" aria-hidden="true">
    <div class="modal-dialog modal-lg" role="document">
        <div class="modal-content">
            <div class="modal-header">
                <h5 class="modal-title" id="paymentModalLabel{{ course.id }}">
                    Payment Receipt - {{ course.title }}
                </h5>
                <button type="button" class="close" data-dismiss="modal" aria-label="Close">
                    <span aria-hidden="true">&times;</span>
                </button>
            </div>
            <div class="modal-body text-center">
                
                    <div class="embed-responsive embed-responsive-1by1 mb-3">
                        <iframe class="embed-responsive-item" 
                                src="{{ url_for('static', filename=course.payment_receipt_path) }}">
                        </iframe>
                    </div>
                
                    <img src="{{ url_for('static', filename=course.payment_receipt_path) }}" 
                         class="img-fluid" alt="Payment Receipt">
                
            </div>
            <div class="modal-footer">
                
                    <form action="{{ url_for('admin.verify_teacher_payment', course_id=course.id) }}" method="POST">
                        <input type="hidden" name="csrf_token" value="{{ csrf_token() }}">
                        <button type="submit" class="btn btn-success">
                            <i class="fas fa-check-circle mr-1"></i> Verify Payment
                        </button>
                    </form>
                
                <button type="button" class="btn btn-secondary" data-dismiss="modal">Close</button>
            </div>
        </div>
    </div>
</div>
\end{lstlisting}

\subsection{Implementation Challenges and Solutions}

The development process encountered several technical challenges:

\subsubsection{File Upload Security}

Secure handling of payment receipts required multiple security measures:
\begin{itemize}
    \item Filename sanitization using secure\_filename
    \item File type validation with FileAllowed validators
    \item Timestamped filenames to prevent collisions
    \item Format-specific rendering (PDF vs. images)
\end{itemize}

\subsubsection{Modal Dialog Functionality}

Modal dialogs initially failed to display correctly across browsers, requiring:
\begin{itemize}
    \item Switching from Bootstrap 5 to Bootstrap 4 for compatibility
    \item Adding explicit JavaScript initialization for modal triggers
    \item Using proper data-toggle and data-target attributes
\end{itemize}

\begin{lstlisting}[language=JavaScript, caption=Modal Initialization, label=lst:modal-js]
$(document).ready(function() {
    $('[data-toggle="modal"]').on('click', function() {
        var target = $(this).data('target');
        $(target).modal('show');
    });
});
\end{lstlisting}

\subsubsection{Subscription Capacity Monitoring}

Tracking student counts across courses required a specialized approach to gather unique students per subscription plan:

\begin{lstlisting}[language=Python, caption=Capacity Monitoring, label=lst:capacity-monitoring]
def get_subscription_usage(teacher_id):
    usage_data = {}
    courses = Course.query.filter_by(teacher_id=teacher_id, is_active=True).all()
    
    # Group courses by subscription plan
    plan_courses = {}
    for course in courses:
        if course.subscription_plan_id not in plan_courses:
            plan_courses[course.subscription_plan_id] = []
        plan_courses[course.subscription_plan_id].append(course)
    
    # Count unique students per plan
    for plan_id, courses in plan_courses.items():
        plan = TeacherSubscriptionPlan.query.get(plan_id)
        if not plan:
            continue
        
        student_ids = set()
        for course in courses:
            enrollments = Enrollment.query.filter_by(
                course_id=course.id,
                payment_verified=True
            ).all()
            student_ids.update(e.student_id for e in enrollments)
        
        usage_data[plan_id] = {
            'plan': plan,
            'current_students': len(student_ids),
            'max_students': plan.max_students,
            'percentage': (len(student_ids) / plan.max_students) * 100 
                if plan.max_students > 0 else 0
        }
    
    return usage_data
\end{lstlisting}